\documentclass{article}
\usepackage[utf8]{inputenc}
\thispagestyle{empty}
\begin{document}
\bibentry{villani2017single}
%\bibliographystyle{plain}
%\bibliography{ref}
  
\medskip

\paragraph{}
Villani \textit{et.al.} demonstrated the use of Seurat to discover new type of Dendritic cells(DCs) and monocytes from the single-cell RNA sequencing (scRNA-seq) data of a healthy donor's blood sample. This paper highlighted the remarkable capability of scRNA-seq in characterizing cell type, in conjunction with using traditional approaches: molecular markers,functional properties and ontogeny. With a more comprehensive gene expression profiles at single-cell level, Villani's team is able to distinguish six DC population and four monocytes populations from the the purified DCs and monocytes. Similar to the previous studies, authors first utilize fluorescence-activated cell sorting (FACS) to isolate the targeted populations based on carefully selected gene markers, followed by deep scRNA-seq. A unsupervised analysis is then carried out to group cells into distinct clusters in which gene expression profile is obtained in each cluster. The classification of the clusters is further refined by the close examination of variable genes among the clusters. With this approach, in addition to provide a more refined cell classification, the team is able to discover a rare cell type, AXL+SIGLEC6+ cells(AS DCs),in the fifth DC cluster in which its gene expression processes a spectrum of variation. However, as the authors point out, some subtypes of cells can be still be missed by various reasons, such as having only non-RNA molecule identifiers or only presented in certain physiological states. A more sophisticated approach is necessary to address those shortcomings.


\end{document}